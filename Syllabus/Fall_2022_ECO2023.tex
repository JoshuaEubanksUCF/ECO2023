\documentclass[11pt]{paper}
\usepackage{geometry}
\usepackage{hyperref}
\usepackage{xcolor}
\geometry{
  top = 1in
  , bottom = 1in
  , left = 1in
  , right = 1in
  }
\hypersetup{
	colorlinks=true,
	linkcolor=blue,
	filecolor=magenta,
	urlcolor=cyan,
}

\begin{document}
\title{ECO 2023 Fall 2022: Principles of Microeconomics}
\author{University of Central Florida --- Department of Economics}

\maketitle
\hrulefill

\section*{Course Information}
\begin{flushleft}
\begin{tabular}{| l | l |}\hline
 Course & ECO 2023 Sections 0001 - 0006 \\\hline
 Term & Fall 2022 \\\hline
 Meeting Location & BA1 0107\\\hline
 Credit Hours & 3 \\\hline
\end{tabular}
\end{flushleft}

\section*{Instructor Information}
\begin{flushleft}
\begin{tabular}{| l | l |}\hline
 Instructor & Joshua Eubanks M.S. \\\hline
 Office & BA2 302Y \\\hline
 Hours & M - Th 9:00am - 11:00am \\\hline
 Email & Email through Webcourses\\\hline
\end{tabular}
\end{flushleft}

\section*{Important Dates}

\begin{itemize}
  \item \textbf{Exam 1}: Oct 03, 2022 7:00am -- Oct 04, 2022 11:59pm
  \item \textbf{Exam 2}: Nov 14, 2022 7:00am -- Nov 15, 2022 11:59pm
  \item \textbf{Final Exam}: Dec 08, 2022 7:00am -- Dec 09, 2022 11:59pm

\end{itemize}

The other important dates are the in-class dates. These depend on your section.

\begin{flushleft}
\begin{tabular}{ c  c  c }\hline
 In-Class Activity & Sections 0001, 0002, 0003 & Sections 0004, 0005, 0006 \\\hline
 1 & Sep 06, 2022 & Sep 08, 2022 \\
 2 & Sep 20, 2022 & Sep 22, 2022\\
 3 & Sep 27, 2022 & Sep 29, 2022\\
 4 & Oct 25, 2022 & Oct 27, 2022\\
 5 & Nov 08, 2022 & Nov 10, 2022 \\\hline
\end{tabular}
\end{flushleft}




% \subsection*{Teaching Assistant Information - TBD }
% \begin{flushleft}
% \begin{tabular}{| l | l |}\hline
%  Office & BA2 302Y \\\hline
%  Hours & TBD \\\hline
% Zoom Hours & TBD \\\hline 
% Email & Email through Webcourses\\\hline
% \end{tabular}
% \end{flushleft}

% \subsection*{Teaching Assistant Information - TBD }
% \begin{flushleft}
% \begin{tabular}{| l | l |}\hline
%  Office & BA2 302Y \\\hline
%  Hours & TBD \\\hline
% Zoom Hours & TBD \\\hline 
% Email & Email through Webcourses\\\hline
% \end{tabular}
% \end{flushleft}

% \subsection*{Teaching Assistant Information - TBD }
% \begin{flushleft}
% \begin{tabular}{| l | l |}\hline
%  Office & BA2 302Y \\\hline
% Hours & TBD \\\hline
% Zoom Hours & TBD \\\hline 
% Email & Email through Webcourses\\\hline
% \end{tabular}
% \end{flushleft}

\newpage

\tableofcontents

\newpage

\section{Course Description}

Economics is the science concerned with the allocation of scarce resources to satisfy the needs and desires of society (consumers, producers, and governments). In this course we will focus upon micro-level aspects of economic behavior and activity. After coverage of the fundamentals related to the emergence and functioning of markets and the economic gains resulting from trade, we will develop a framework to describe and evaluate the ‘optimizing’ behavior of the micro-economic entities that comprise markets: buyers (individuals and households) and sellers (resource owners, producers, and service providers).\\

This course aims to build your critical thinking skills and your communication skills – these are the two skills employers are looking for in their employees because these are the skills that are the most difficult to automate. The online component of the course will introduce you to key microeconomic concepts and provide you with opportunities to apply what you are learning.

\subsection{Course Modality}

This is a \textbf{REAL} (\underline{R}elevant, \underline{E}ngaged, \underline{A}ctive \underline{L}earning) course. REAL classes are not traditional ``in person'' classes. The majority of the class content is provided online (text, lectures, homework, etc). In addition to the online content, there are 6 \textbf{in-person} labs scheduled over the semester. The first session is an introduction/orientation for the course and reviews class policies. The remaining 5 labs are similar to traditional science labs. Students gather in groups to work on an assignment that is submitted at the end of the lab. These sessions are not lectures and new content is not presented. The role of the instructor and TAs who are present during the active learning labs is to answer questions about the specific lab assignment and proctor the session. More information about this modality can be found \href{https://youtu.be/vuOArqNs4mQ}{here}.

\subsection{Enrollment Requirements}
There are no prerequisites for this course.  You need an open mind, intellectual curiosity, a desire to learn, and a willingness to participate in the learning process.  You are expected to remember basic grade school (K – 12) level math concepts.\\

REAL courses push you out of your comfort zone of being a student and into a more accurate representation of being a professional. In order to have a successful experience in this class, you need to be self-motivated and prepared to complete work on your own time. You need to be able to manage your time well to avoid falling behind or forgetting to complete assignments. \textbf{You are not alone in this learning endeavor}. When attending my office hours or making an appointment to meet with me, you will discover you have my undivided attention. My TAs are ready to help you as well. Here are some \href{https://learningcenter.unc.edu/tips-and-tools/motivation/}{tips} to get you be self-motivated. Additional content to be a better overall student can be found \href{https://learningcenter.unc.edu/tips-and-tools/}{here}.

\section{Student Learning Outcomes}
By the end of this course, you should be able to:

\begin{enumerate}
	\item Recognize why scarcity of resources necessitates that choices must be made, and thus why trade-offs occur when decisions are made;
  	\item Differentiate between comparative and absolute advantage, and explain why the former is the key principle behind gains from trade;
	\item Use marginal cost/marginal benefit analysis to determine behavior for optimal decision making;
	\item Recognize how incentives affect behavior;
	\item Determine the market clearing price and quantity in a market, and predict the effect on the equilibrium price and quantity when the factors that shift a demand or supply curve change;
	\item Explain why competitive markets lead to economically efficient outcomes;
	\item Use demand/supply analysis to assess the economic implications of government interventions, including concepts of consumer/producer surplus and deadweight loss;
	\item Calculate elasticities and their impact on firms’ decisions and market outcomes;
	\item Determine the profit maximizing price and output decisions of a perfectly competitive firm in the short run and the long run; and
	\item Determine the profit maximizing price and output decisions of firms with market power in the short-run and the long-run
\end{enumerate}

\section{Course Materials}
 
There are some items that you will need to purchase in order to pass the course. Without it, you will miss assignments that are 13\% of your grade. You will also have trouble passing this class without the practice and readings from the textbook.\\

Besides the textbook, you will need the following items:

\begin{itemize}
	\item A \textbf{non-programmable} calculator. All exams will be in the testing center. Programmable calculators are not allowed the testing center. 
	\item A reliable internet connection. My content will be mostly videos, you will need a decent internet connection in order to watch the videos
	\item An Andriod or iPhone device with QR code scanning abilities. This is how we will take attendance. More information is provided in section \ref{sec:ucfhere}
\end{itemize}

\subsection{Textbook}

The required textbook is \textbf{Asarta / Butters 3e | Connect Master for Principles of Economics - McGraw-Hill. ISBN: 97812643222121}. You will be able to access this material through FirstDay\texttrademark.\\

\subsubsection{FirstDay\texttrademark}

To enhance your learning experience and provide affordable access to the right course material, this course is part of an inclusive access model called FirstDay\texttrademark. You can easily access the required materials for this course at a discounted price, and benefit from single sign-on access with no codes required in UCF Webcourses. You can learn more about FirstDay\texttrademark \href{https://video.mhhe.com/watch/KuB92tcunQc2TisgtWgKAE?}{here}.\\

It is recommended that you Opt-In as these materials are required to complete the course. You can choose to Opt-In on the first day of class, right within UCF Webcourses. Be sure to Opt-In before the deadline of Friday, August 28, 2022 at 11:59pm to have access to your course materials at the discounted price. If you do not opt-in by August 28, you’ll have to pay full price and may have delays accessing the course assignments. When you opt-in, your \textit{UCF Student Account} will be charged directly. Note: the compatible web browsers are Google Chrome, Firefox, Internet Explorer. Safari is \textbf{NOT} compatible\\

Here are some resources for support. If you need to contact someone, refer to section \ref{sec:toolsupport}

\begin{itemize}
\item Opting-In and Accessing your eTextbook: \href{https://vimeo.com/306061595}{https://vimeo.com/306061595};
\item Opting-In for your Courseware Materials: \href{https://vimeo.com/304673669}{https://vimeo.com/304673669};
\item Other Features on the Course Materials page: \href{https://vimeo.com/304675344}{https://vimeo.com/304675344};
\item First Day FAQ: \href{https://tinyurl.com/UCF-FirstDay-FAQ}{https://tinyurl.com/UCF-FirstDay-FAQ};
\end{itemize}


\section{Course Structure}

I will update the home page each Monday morning containing the content that must be completed before the following week.\\

The tentative schedule is as follows:   

\begin{center}
\begin{tabular}{| l | l |}\hline
 Week Start & Material  \\\hline 
 Aug 22 & REAL Class Introduction, Fundamentals \\
 Aug 29 & Demand \\
 Sep 05 & REAL Class Activity\\
 Sep 12 & Supply\\
 Sep 19 & REAL Class Activity\\
 Sep 26 & REAL Class Activity\\
 Oct 03 & Market Equilibrium and Policy\\
 Oct 10 & REAL Class Activity\\
 Oct 17 & Market Efficiency\\
 Oct 24 & REAL Class Activity\\
 Oct 31 & Elasticity\\
 Nov 07 & REAL Class Activity\\
 Nov 14 & Production and Perfect Competition\\
 Nov 21 & Perfect Competition\\
 Nov 28 & Pure Monopoly\\\hline
\end{tabular}
\end{center}

\section{Assessment and Grading Procedure}

Below is a table consisting of all the activities you will be graded on. 

\begin{flushleft}
\begin{tabular}{ l  l }\hline
Activity & Percent of Grade \\\hline
Proof of Enrollment Assignment &  0.5\% \\
Application Based Activities & 2.5\% \\
LearnSmart Assignments & 2.5\% \\
Homework & 7\% \\
In-Class Assignments & 12.5\% \\
Midterm Exam 1 & 24\% \\
Midterm Exam 2 & 24\% \\
Final Exam & 27\% \\\hline
\end{tabular}
\end{flushleft}

In all of the McGraw-Hill Connect Assignments, the lowest two scores will be dropped. All assignments in each group are equally weighted regardless of the number of questions in each assignment.

\subsection{Proof of Enrollment Assignment (0.5\%)} \label{sec:fasfaquiz}

As outlined in section \ref{sec:fasfa}, there must be an assignment to prove your enrollment. Our proof of enrollment is through an online quiz in Webcourses. The quiz covers the syllabus and general information (orientation module). In order to document that you began this course, please complete this Quiz by Friday August 26, at 5pm. Late submissions may result in a delay in the disbursement of your financial aid. You will have one hour to complete the 20 questions and it can be taken only once.\\

Access to the correct answers will be available after the due date. To receive an explanation of the correct answers please contact the TAs or myself.

\subsection{Application Based Activity (2.5\%)}

These assignments provide students with valuable practice using problem solving skills to apply their knowledge to realistic scenarios. They will be completed on-line on McGraw-Hill Connect. Each assignment can be taken three times. The system will keep your best score.


\subsection{LearnSmart Assignments (2.5\%)}

There will be 9 assignments in total.

\subsection{Homework (7\%)}

The purpose of the homework assignments is to help you assess how well you understand the course material. Homework assignments are open book and open notes.

\begin{itemize}
\item After the first 2 attempts Students will see their total scores, question responses with scores, and correct or incorrect indicators
\item After submitting the third attempt, students will see their total scores, question responses with scores, correct or incorrect indicators, explanations, and solutions. After the due date, students will be able to view assignments without affecting their grade. Study attempts are allowed.
\end{itemize}

\subsection{In-Class Assignments (12.5\%)}

Dates of the in-class sessions depends on your section:\\

\begin{flushleft}
\begin{tabular}{ c  c  c }\hline
 In-Class Activity & Sections 0001, 0002, 0003 & Sections 0004, 0005, 0006 \\\hline
 1 & Sep 06, 2022 & Sep 08, 2022 \\
 2 & Sep 20, 2022 & Sep 22, 2022\\
 3 & Sep 27, 2022 & Sep 29, 2022\\
 4 & Oct 25, 2022 & Oct 27, 2022\\
 5 & Nov 08, 2022 & Nov 10, 2022 \\\hline
\end{tabular}
\end{flushleft}

After Orientation at the First Day of Class session, you are expected to attend 5 Active Learning Lab (ALL) sessions during the semester. Attendance is taken in each of these ALL sessions. If you miss an ALL session, for any reason, or if you want to improve your grade of one of your in-class LAB grades, you can take the LAB makeup.

\begin{itemize}
\item There are 5 LABS and 1 LAB Makeup. The best 5 assignments will count toward your final grade
\item The makeup LAB assignment is open for the entire class. Your score on the makeup lab will substitute one of the 5 scores of the Active Learning Activity - only if you do better on the makeup. If you do not take the makeup or if you do worse than one of the 5 assignments the score will be dropped in Webcourses and will not count towards your final grade.
\item You can only take one makeup unless all of your absences are excused based on the UCF policy for absences.
\item Purpose: The ability to work in teams and to clearly communicate information are two important skills required by employers. Problem Solving Teams will allow you to develop these skills. Group interaction, exchanging and discussing ideas, even teaching one another are all encouraged and expected. Individuals coming to class unprepared and expecting their teammates to pick up their slack is not acceptable. If the majority of the students in a team think that one of the teammates is not contributing to the in class activities, I reserve the right to deduct up to 100\% of the points for that particular assignment for that student.
\item After the drop date you will be assigned to a team at random. Each team will consist of 6-8 members who will work collectively to solve problems in class. Yes, I want you to talk to your team-mates and help each other. I suggest you exchange phone numbers.
\item To find your team/group: in Course Navigation, click the People link, then click the Groups tab.
\item There will be an assignment due from each team member at the end of each class meeting. One paper per group will be selected randomly for grading and the entire group will receive the same grade.

\item You must sign capture your attendance using UCFHere. If you do not have a phone, you must write your name on a sheet of paper. More information is provided in section \ref{sec:ucfhere}
\item Like in all REAL classes, If you are 10 minutes and 1 second late to the Active Learning Lab you will receive a 0 for the in class activity.
\item You must each print your names (Print: First Name, Last Name) on your submitted assignment. If there is no name or no team number or no table number you will receive a 0 for the in class assignment.
\item Please use a pencil. Illegible and messy papers will receive a zero.
\item The assignments will be equally weighted even though some assignments have more questions than others. 
\item Grades will be uploaded 2 weeks after each in class meeting. If your grade is not up at the end of the second week you must contact my TA’s. At the end of the semester I won’t look for assignments done at the beginning of the semester.
\item If you are happy with the scores of your 5 in class LABs you don’t have to take the makeup. However if you missed a LAB or if you are not happy with one of your in class scores you should take the makeup assignment. The score on the makeup lab will substitute one of the 5 in class scores of the Active Learning Activity - only if you do better on the makeup. If you don’t take the makeup or if you do worse than one of the 5 assignments the score will be dropped in Webcourses and will not count towards your final grade.
This will be an online, cumulative assignment covering content from the entire semester. You will have 80 minutes to complete the assignment. You can take the assignment through Webcourses (link is provided) from anywhere. You must take the assignment during the scheduled date/time. No exceptions will be allowed. Don’t forget, only the best 5 scores will count toward your final grade
\item Orientation + Active Learning Lab meetings - A few rules to keep in mind:
\begin{enumerate}
\item You MUST attend the class section for which you are registered. No exceptions will be made. There are 6 sections of ECO 2023. If you don’t know your section, you can check it under the People tab in Webcourses for your specific class time. The 6 class dates are also posted in Webcourses and at the end of the syllabus.
\item Show up on time. Students must be seated and ready to begin work at the start of class time. Tardiness will be dealt with in the following manner: If you are 10 minutes and 1 second (or more) late to the active learning lab you will receive a 0 for the in class activity. I reserve the right to not allow you entry into the class after 10 minutes.
\item Do not plan to leave early. You are required to stay for the entire session and work with your group. I reserve the right to deduct up to 100\% from your assignment for that day if you leave early.
\item Come to class prepared. You are expected to have not only completed the assignments due in the weeks prior to the class meeting, but you are also expected to have reviewed the material so you can be a productive member of your group.
\item If my TAs or I find that one of the group members is not participating with his/her group we will take up to 100\% off the score for that particular in class activity
\end{enumerate}
\end{itemize}

\subsection{Exams}

\begin{itemize}

\item All exams dates have been determined. Be sure to keep these dates in mind.

\begin{itemize}
  \item \textbf{Exam 1}: Oct 03, 2022 7:00am -- Oct 04, 2022 11:59pm
  \item \textbf{Exam 2}: Nov 14, 2022 7:00am -- Nov 15, 2022 11:59pm
  \item \textbf{Final Exam}: Dec 08, 2022 7:00am -- Dec 09, 2022 11:59pm
\end{itemize}

\item You must take these exams in the Keon Testing Center. There are very specific rules about the center. Please read the \href{https://business.ucf.edu/centers-institutes/keon-testing-center/ktc-student-guide/}{Keon Center Student Guide.}

\item If you are escorted out of the testing center or if you are denied entrance for any testing center violation, I won’t reopen the test. You should bring your own non-programmable calculator.

\item Not all questions are multiple choice. The tests will have a mix of conceptual and numerical questions. The numerical questions will be similar to the problems you do for homework and the example problems in each chapter and notes. 

\item A formula sheet will be embedded in each test at the beginning of the test in Canvas. Please be aware notecards will not be allowed in the testing center.

\end{itemize}

\subsubsection{Two Midterm Exams (24\% Each)}

There will be 2 midterm exam totalling 48\% of your grade. Both tests will have 30 questions. 

\begin{itemize}
\item You will have 80 minutes to complete each test. Each test is taken at the testing center.
\item The test questions will not be available to you after you finish taking each test. You will only have access to see the grade you earned. If you want to go over the questions you missed be sure to contact the TAs.
\item You will have up to 2 weeks after the closing of each test to go over the questions you missed.

\end{itemize}
\subsubsection{Final Exam (27\%)}


The final exam is mandatory and cumulative - total of 27\% available. It consists of 40 questions. You will have 110 minutes to complete the final exam.

\subsection{Grading Scale}

All grades will be rounded to the nearest tenth of a percent.

\begin{flushleft}
\begin{tabular}{ l  l }\hline
 Percent & Grade \\\hline 
 $\geq$ 92.0 &  A \\
 91.9-89.5 & A-\\
 89.4-87.0 & B+ \\
 86.9-82.0 & B \\
 81.9-79.5 & B- \\
 79.4-76.0 & C+ \\
 75.9-69.5 & C \\
 69.4-66.0 & D+ \\
 65.9-59.5 & D \\
 $\leq$59.4 & F \\\hline
\end{tabular}
\end{flushleft}

\subsubsection{Reporting}

Grades will be added via Webcourses to follow student data classification and security standards. We are not allowed to discuss grades outside of Webcourses. Additionally, your grades are protected by FERPA. I cannot discuss your grades with anyone besides you. 

\section{Help}

The minimum face-to-face interaction you will have is in our 6 meetings. I will however keep in contact in the following ways:

\begin{itemize}
	\item Announcements and emails every Monday ensuring that you are aware of what is expected to be completed that week.
	\item Every other week, I will check your grade in the class. If you are below a 70\%, I will reach out and ask how my TAs and I can be of assistance
\end{itemize}

If you would like to meet in-person outside the lab sessions, you can meet with my TAs or I during office hours. 

\subsection{In-Person Office Hours}

All of our office hours will be posted on the Webcourses homepage. If you are meeting during those times, you do not need to make an appointment.

\subsubsection{Instructor Office Hours}

I will hold office hours and you can feel free to stop by. We can review concepts from class or work through an example. We can chat about your study strategy and progress in the course. Additionally, if you’ve read an interesting article or heard something in the news that relates to our class, I am happy to discuss those items as well. If you want to meet at a different time, please email me to schedule an appointment. 

\subsubsection{Teaching Assistants Office Hours}

We will have 4 teaching assistants available if you need assistance. Be sure to utilize the asisstants with any questions you may have. They can help you with homework assignments, prepping for an exam, and reviewing previous exams. 

\subsection{Online Assistance}

If you cannot meet in person for any reason, here are some virutal ways of getting help.

\subsubsection{Virtual Office Hours}

We will have virtual hours posted on the home page as well. To join, you can click on the "Zoom" tab on the lefthand side of Webcourses. This will be a group meeting where each student will be able to ask questions one at a time in the order of arrival. If you want to review your grades, you will \textbf{not} be able to do so in this forum. You will need to meet in-person to discuss grades on assignments/exams.

\subsubsection{Discussions}

I find it useful to use the discussion board in webcourses to communicate with students about the course content. I will create topics for each module and exam.  If you are struggling with a concept or require explanation about a homework or exam question, post your question under the appropriate topic and someone will answer. Please try to refrain from posting questions that have already been asked and answered.  If you would like to answer another student’s question, please feel free but be sure to post an \textbf{explanation} and not just an answer. The teaching assistants will also be moderating the discussion board, so please refrain from inappropriate language, advertising, and other violations of \href{https://www.rasmussen.edu/student-experience/college-life/netiquette-guidelines-every-online-student-needs-to-know/}{netiquette}. If there is inappropriate content, I reserve the right to deduct points from your overall grade.

\subsubsection{Webcourses Email}

If your question isn't already answered in the discussions, feel free to email us through Webcourses only. Please copy the TAs and I in all of your communications. We will respond within \textbf{2} business days. If you want to discuss grades (or other personal matters), feel free to send those messages directly to me.

\subsubsection{Online Resources Support} \label{sec:toolsupport}

If it is a technology based issue, please reach out in the following ways:

\begin{itemize}

\item \textbf{FirstDay\texttrademark}
	\begin{itemize}
	\item Creating a support ticket online: \href{https://tinyurl.com/customercarerequest}{https://tinyurl.com/customercarerequest}; and
	\item Emailling the Customer Care Team Directly \href{mailto:bookstorecustomercare@bncollege.com}{bookstorecustomercare@bncollege.com};
	\end{itemize}

\item \textbf{McGraw Hill Connect}
	\begin{itemize}
	\item Contact the Customer Experience Group (GXG) at \href{tel:800-331-5094}{800-331-5094}. They will create a case number associated with the issue so that we can follow up on it in the future if needed.
	\end{itemize}

\item \textbf{Webcourses Support}
	\begin{itemize}
	\item If you have any issues with Webcourses, there are many ways to recieve support. This website will have all the available options: \href{https://cdl.ucf.edu/support/webcourses/}{https://cdl.ucf.edu/support/webcourses/}
	\end{itemize}

\end{itemize} 


\subsubsection{External Social Media Groups}

I understand social media platforms such as GroupMe can be more convenient than the discussion board as far as getting help from a fellow student.  An advantage to the discussion board is the TAs and I are the experts as far as the course content is concerned. That being said, if you want to communicate with other students using GroupMe, etc. you are free to do so but I cannot take responsibility for anything said in the group.  Additionally, it is a violation of my policies and UCF’s Golden Rule for you to post homework, quiz, or exam questions/answers on any online site. You may discuss concepts and procedures for solving problems as long as you do not explicitly post questions/answers. Thanks in advance for your cooperation and if you feel other students are in violation of my policies or UCF’s Golden Rule, please feel free to bring it to my attention.

\subsection{Student Academic Resource Center (SARC) Tutoring Service}

At the time of the syllabus generation, there is not a student assigned to this course. Please refer to \href{https://ucfsarc.wordpress.com/eco-2023/}{https://ucfsarc.wordpress.com/eco-2023/} for any updates

\subsection{Non-Course Related Assistance}

If you have any other issues, non-related to the course, here are some of the resources available.

\subsubsection{UCF Cares}

UCF Cares is an umbrella of care-related programs and resources dedicated to fostering a caring community of Knights. UCF Cares Focus Areas include:
\begin{itemize}
\item Safety and Wellbeing Services
\item Violence Prevention
\item Mental Health Support
\end{itemize}

Check out this link for more information and assistance \href{https://cares.sdes.ucf.edu/}{https://cares.sdes.ucf.edu/}


\subsubsection{Let's Be Clear}

You can find help and support here for sexual harassment, sexual assault, relationship violence and stalking. Check out this link for more information \href{https://letsbeclear.ucf.edu/}{https://letsbeclear.ucf.edu/}

\section{Policy Statements}
\subsection{Academic Integrity}
The Center for Academic Integrity (CAI) defines academic integrity as a commitment, even in the face of adversity, to five fundamental values: honesty, trust, fairness, respect, and responsibility. From these values flow principles of behavior that enable academic communities to translate ideals into action.\footnote{\url{https://academicintegrity.org/}}\\

The \href{https://osrr.sdes.ucf.edu/}{Office of Student Rights and Responsibilities}\footnote{Located in Ferrell Commons, Room 227}will be notified of any instance of academic misconduct that has occurred inside or outside of the classroom. Students are encouraged to read the \href{https://goldenrule.sdes.ucf.edu/}{Golden Rule Student Handbook}.\\

Students should familiarize themselves with UCF’s Rules of Conduct. According to Section 1, ``Academic Misconduct,'' students are prohibited from engaging in
\begin{enumerate}
\item Unauthorized assistance: Using or attempting to use unauthorized materials, information or study aids in any academic exercise unless specifically authorized by the instructor of record. The unauthorized possession of examination or course-related material also constitutes cheating.
\item Communication to another through written,visual,electronic, or oral means:The presentation of material which has not been studied or learned, but rather was obtained through someone else’s efforts and used as part of an examination, course assignment, or project.
\item Commercial Use of Academic Material: Selling of course material to another person, student, and/or uploading course material to a third-party vendor without authorization or without the express written permission of the university and the instructor. Course materials include but are not limited to class notes, Instructor’s PowerPoints, course syllabi, tests, quizzes, labs, instruction sheets, homework, study guides, handouts, etc.
\item Falsifying or misrepresenting the student’s own academic work.
\item Plagiarism: Using or appropriating another’s work without any indication of the source, thereby attempting to convey the impression that such work is the student’s own.
\item Multiple Submissions: Submitting the same academic work for credit more than once without the express written permission of the instructor.
\item Helping another violate academic behavior standards.
\end{enumerate}
For more information about plagiarism and misuse of sources, see ``Defining and Avoiding Plagiarism: The WPA Statement on Best Practices.''\\


UCF faculty members strive to provide a quality education, and so seek to prevent unethical behavior and when necessary, respond to infringements of academic integrity. Penalties can include a failing grade in an assignment or course, suspension/expulsion from the university, and/or a ``Z'' designation\footnote{More information on Z designation \href{https://goldenrule.sdes.ucf.edu/zgrade}{here}.} on a students official transcript designating academic dishonesty. 

\subsection{Active Duty Military}
Students under active duty in the military will be accommodated as much as possible. Please see me  prior to scheduled military obligations if this applies to you.

\subsection{Attendance/Late Policies}

Late work will not be accepted. No exceptions.\\

Attendance is recorded for our in-class sessions. If you are more than 10 minutes late, you will be counted as absent and you will receive a 0 on the assignment. You are welcome to stay and work on the assignment, just know that you will not recieve credit for the assignment. Attendance is captured using a mobile application called \textbf{UCF Here}.

\subsubsection{UCF Here} \label{sec:ucfhere}

This attendance capturing measure cuts back on manual work and provides a digital record of your attendance. You will need to bring your phone to class and have the application installed on your \href{https://play.google.com/store/apps/details?id=edu.ucf.ucfhere&hl=en_US}{Android} or \href{https://itunes.apple.com/us/app/ucf-here/id1450015124}{iPhone} device. UCF Here does not work on tablets or laptop computers.\\

Here is the \href{https://ucf.service-now.com/kb_view.do?sysparm_article=KB0013581}{UCF Here Student Guide} for further information.

\subsection{Emergency Procedure and Campus Safety}
Be aware of your surroundings and be familiar with the necessary actions to take in the event of an emergency. In case of an emergency, dial 911 for assistance. All classrooms contain an emergency procedure guide and is available \href{http://emergency.ucf.edu/emergency_guide.html}{online}. I advise signing up for text alerts from UCF if not already registered. Steps are below:
\begin{itemize}
	\item Log in to myUCF
	\item Click the `Student Self Service' tab
	\item Click the `Personal Information' tab
	\item Click the `UCF Alert' tab
\end{itemize} 

If there is a medical emergency during class, students may need to access a
first-aid kit or AED (Automated External Defibrillator). Here is the \href{http://www.ehs.ucf.edu/AEDlocations-UCF}{link} to learn where those are located.\\

To learn about how to manage an active-shooter situation on campus or elsewhere, consider viewing this \href{https://www.youtube.com/watch?v=NIKYajEx4pk&feature=youtu.be}{video}.\\

Students with special needs related to emergency situations should speak with me outside of class.
\subsection{Extra Credit}

I do not provide any extra credit. I provide additional ways to increase your grade in the course: dropped assignments and an extra in-class exercise. Each student has incurred opportunity costs when studying for this course. It would be unfair for me to bump a student to an ``A'' when another student spent more time on the course to ensure they recieve an ``A.'' Any emails/conversations about extra credit or bumping your grade to the next level will be ignored.

\subsection{Make-up Exams and Assignments}
Per university policy, students may only turn in make-up work (or and equivalent, alternate assignment) for \textbf{university-sponsored events, religious observances, or legal obligations (i.e. jury duty)}. In these instances, students are excused without penalty.\\

Students who know they will be absent due to a religious observance must notify me at the beginning of the semester so that make-up work can be arranged. For more information, please refer to the \href{https://regulations.ucf.edu/docs/notices/5.020ReligiousObservancesNEW_Oct09_000.pdf}{policy}.\\

If you miss one of the two midterm tests for any reason you will be given the opportunity to use the Final exam in-place of the missed midterm.\\

In terms of assignments, you are provided with sufficient time to complete each assignment. If you miss the assignment, you will recieve a zero.

\subsection{Revisions to the Syllabus}
The contents of this syllabus may change as we progress through the semester. I will post an announcement declaring any major changes to the syllabus through Webcourses.
\subsection{Student Academic Activity Policy} \label{sec:fasfa}

Since Fall 2014, all faculty members are required to document the student's academic activity at the beginning of the course. The first quiz is proof of academic activity. Please complete the quiz by the first Friday at 5pm, or you may delay/lose your financial aid. More details about the assignment can be found in section \ref{sec:fasfaquiz} 
\subsection{Student Accessibility Services}
The University of Central Florida is committed to providing access and inclusion for all persons with disabilities. Students with disabilities who need disability-related access in this course should contact the lecturer as soon as possible. Students should also connect with \href{https://sas.sdes.ucf.edu}{Student Accessibility Services} located at Ferrell Commons room 185, by \href{mailto:emailsas@ucf.edu}{email}, or phone 407-823-2371. Through Student Accessibility Services, a Course Accessibility Letter may be created and sent to professors, which informs faculty of potential access and accommodations that might be reasonable. Determining reasonable access and accommodations requires consideration of the course design, course learning objectives and the individual academic and course barriers experienced by the student.
\end{document}