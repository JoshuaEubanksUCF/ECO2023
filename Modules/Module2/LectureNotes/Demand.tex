\documentclass{beamer}
\usetheme{PegasusUCF}


%-------------------------------
\title[Demand]{Demand}
\author{Joshua L. Eubanks}

\newcommand{\deptname}{12}%ECONOMICS

%<<<<<
\date{\vspace{-8cm}}

%--------------------------------



%%%%%%%%%%%%%%%%
\begin{document}
%%%%%%%%%%%%%%%
\begin{frame}
  \titlepage
\end{frame}


\begin{frame}{Overview}
\tableofcontents
\end{frame}

\begin{frame}{Note About Blanks Slides}

\begin{alertblock}

If you see a blank slide, it is intentional.  All graphs will be provided within my videos. 
\end{alertblock}

\end{frame}

\section{Introduction to Markets and Prices}

\begin{frame}{Introduction to Markets and Prices}

\begin{block}{Market}
Any place where, or mechanism by which, buyers and sellers interact to trade goods, services, or resources.
\end{block}

Prices are determined by the interaction of buyers and sellers.

\begin{alertblock}

\begin{itemize}
\item Sellers do not set prices
\item Buyers do not set prices
\end{itemize}
\end{alertblock}

Take oranges for example, if the price is too high, people will not buy them. If they are too low, sellers will not sell them. This is where terms of trade re-enters the conversation

\end{frame}

\begin{frame}{Price Analogy}

\begin{exampleblock}

A famous analogy is that prices are like scissors. 

\begin{itemize}

\item You might be tempted to ask the question, who sets the price? The buyer or the seller?
\item This would be like asking, which blade of the scissors cuts the paper? The top or bottom blade?
\item It is in-fact both that cut the paper and the interaction between both buyers and sellers that determine the price
\end{itemize}

\end{exampleblock}

\end{frame}



\section{Demand Curve}

\begin{frame}{Law of Demand}

\begin{block}{Law of Demand}
A principle in economics which states that as the price of a good, service, or resource rises, the quantity demanded will decrease, and vice versa, all else held constant.
\end{block}

\end{frame}

\begin{frame}{Demand Schedule}

Just like the production possibilities schedule, this is a tabular view of the quantity demanded for a given price

\begin{exampleblock}{Demand Schedule for Movie Tickets}

\begin{center}
\begin{tabular}{  c  c  }\hline
 Price & Quantity Demanded \\\hline
16 & 1\\ 
13 & 3\\
10 & 5\\
7 & 7\\
4 & 9\\\hline
\end{tabular}
\end{center}
\end{exampleblock}


\end{frame}

\begin{frame}{Demand for Movies Graphically}
\end{frame}

\section{Law of Demand - Marginal Benefit, Purchasing Power, and Substitutes}

\begin{frame}

\end{frame}

\section{Market Demand}

\begin{frame}

\end{frame}

\section{Changes in Demand}

\begin{frame}

\end{frame}

\section{Determinants of Demand}

\subsection{Income}

\begin{frame}

\end{frame}

\subsection{Tastes and Preferences, Number of
Buyers, and Expectations}

\begin{frame}

\end{frame}

\subsection{Substitutes and Complements}

\begin{frame}

\end{frame}

%%%%%%%%%%%%%%
\end{document}