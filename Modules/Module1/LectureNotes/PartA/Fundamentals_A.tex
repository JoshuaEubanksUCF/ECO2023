\documentclass{beamer}
\usetheme{PegasusUCF}


%-------------------------------
\title[Fundamentals]{Fundamentals Part A}
\author{Joshua L. Eubanks}

\newcommand{\deptname}{12}%ECONOMICS

%<<<<<
\date{\vspace{-8cm}}

%--------------------------------



%%%%%%%%%%%%%%%%
\begin{document}
%%%%%%%%%%%%%%%
\begin{frame}
  \titlepage
\end{frame}

\begin{frame}{Overview}
\tableofcontents
\end{frame}

\section{Introduction to Micro/Macroeconomics}

\begin{frame}{What is Economics?}

Many textbooks will give a very lengthy definition, but here is the most consise definition

\begin{block}{Economics}
The study of choices by individuals, firms, governments make in a world of scarce resources.
\end{block}

You can see that the definition does not include money. Money is simply a tool to make transactions easier.

\end{frame}

\begin{frame}{Microeconomics v. Macroeconomics} 

  \begin{block}{Microeconomics}
The study of the economy at the small-scale level, examining individuals and specific markets.
  \end{block}

  \begin{block}{Macroeconomics}
The study of the economy at the largescale level, examining total output, the price level, and other aggregate measures in the economy.
  \end{block}

A common analogy is that \textbf{Macroeconomics} is the forest, and \textbf{Microeconomics} are the individual trees within the forest. 

\end{frame}

\section{Resources}


\begin{frame}{What are resources?}

Resources are the foundation of all productive activity.

\begin{block}{Resource}
Any item, whether a gift of nature, the result of production, or the result of human effort, that is used to produce goods and services.
\end{block}

\end{frame}

\begin{frame}{Four Categories of Resources}

\begin{itemize}
\item Land
\item Labor
\item Capital
\item Entreprenurial Ability
\end{itemize}

\begin{exampleblock}{Example: Julia's Bakery}
For the next set of slides consider Julia, who just started their own bakery. We will provide examples of each resource used.
\end{exampleblock}

\end{frame}



\begin{frame}{Land}

\begin{block}{Land}
All natural resources, or ``gifts of nature,'' including farmland, forests, oil, oceans, etc.
\end{block}

\begin{exampleblock}{Land: Julia's Bakery}
The wheat she uses in her flour can be considered land. Anything that grows from the land is also categorized as land.
\end{exampleblock}

\end{frame}

\begin{frame}{Labor}

\begin{block}{Labor}
Human effort, both physical and mental
\end{block}

\begin{exampleblock}{Labor: Julia's Bakery}
The effort that Julia uses when icing the cupcakes.
\end{exampleblock}

\end{frame}

\begin{frame}{Capital}

\begin{block}{Capital}
Tools, machinery, infrastructure, and knowledge used to produce goods and services. A key point is that capital does not end up in the good or service itself.
\end{block}

\begin{exampleblock}{Capital: Julia's Bakery}
The oven used by Julia to bake the cupcakes.
\end{exampleblock}

\end{frame}

\begin{frame}{Entreprenurial Ability}

\begin{block}{Entreprenurial Ability}
Talent to combine the land, labor, and capital into a productive process
\end{block}

\begin{exampleblock}{Entrprenurial Ability: Julia's Bakery}
Julia's ability to efficiently produce cupcakes. Note, this is about combining the products efficently, not her knowledge of how to make a cupcake. Her knowledge of how to make a cupcake would be considered \textit{human capital}.
\end{exampleblock}

\end{frame}

\section{Scarcity}


\begin{frame}{What is Scarcity?}

\begin{block}{Scarcity}
A condition that results from the inability of limited resources to satisfy unlimited wants.
\end{block}

\begin{exampleblock}{Example: Time}
A resource that is scarce for everyone regardless of income or wealth.
\end{exampleblock}
\end{frame}

\section{Opportunity Cost}


\begin{frame}{What is Opportunity Cost?}

Frederic Bastiat's suggestion is that to be a good economist, you must train yourself to see the unseen.

\begin{block}{Opportunity Cost}

The value of the next-best forgone alternative; the value of the opportunity that you gave up when you chose one activity, or opportunity, instead of another

\end{block}

\begin{exampleblock}{Example: Farming}
If you are using your land to grow coffee beans, you cannot use that same plot of land for cocoa beans. 
\end{exampleblock}

\end{frame}


\section{Incentives}

\begin{frame}{Role of Incentives}

\begin{itemize}
\item When economists look at the world they believe that people respond to incentives
\item If a behavior receives a reward people will do more of it
\item If you penalize a behavior, people will do less of it
\item  Comprehending the importance of the role
incentives play in how the market works is one
the goal of economics
\end{itemize}

\begin{exampleblock}{Example: Traffic Fines}

If you do not want individuals to speed, you can create penalties to reduce the number of people speeding.
\end{exampleblock}

\begin{alertblock}{Note: Incentives do not always turn out as expected}

When funding of schools were directly linked to test scores, teachers were incentivized to extend exam times, give correct answers, and even complete the tests for their students. 
\end{alertblock}
\end{frame}

\section{Rational Decision Making}

\begin{frame}{What is Rational Decision Making?}

Rational decision making is one of the key assumptions of economics. Without it, we could not make any predictions about how people will respond to changes.

\begin{block}{Rational Decision Making}
Descisions are considered rational based on three characterisitics:
\begin{enumerate}
\item Self interest
\item Involve Marginal Analysis
\item Optimizes the overall well-being of the decision maker
\end{enumerate}
\end{block}

\begin{exampleblock}{Example: Buying Candy}
Suppose you are at the grocery store, you see two identical candy bars. One is priced at \$0.50 and the other is \$1. A rational decision would be to buy the candy bar at \$0.50 since the candy bars are identical.
\end{exampleblock}

\end{frame}

\section{Marginal Analysis}

\begin{frame}{What is Marginal Analysis?}

\begin{block}{Marginal Analysis}
Evaluating the costs and benefits associated with each additional decision.
\begin{itemize}
\item \textbf{Marginal Benefit:} Additional benefit associated with one more unit of activity
\item \textbf{Marginal Cost:} Additional cost associated with one more unit of activity
\end{itemize}

\end{block}

Typically, as your level of activity increases, the marginal benefit decreases and the marginal cost increases

\begin{exampleblock}{Example: Building Libraries}
Consider the decision to the number of libraries you need to build. 
\end{exampleblock}

\end{frame}

\begin{frame}{Building Libraries Example}

Using basic intuition, we know that a library on every corner will be too many, and one library may be too few, but how can we define this?

\begin{exampleblock}{Marginal Benefit}
The first library will provide a lot of benefit to the community. The second one, however, will not provide as much benefit as the second one.
\end{exampleblock}

\begin{exampleblock}{Marginal Costs}
As you build more libraries, your marginal costs will rise. You will have to use more expensive land that has better alternative uses.
\end{exampleblock}

\begin{alertblock}{Food for Thought}
As resources are going digital, how would this impact the optimal number and size of libraries?
\end{alertblock}

\end{frame}

%%%%%%%%%%%%%%
\end{document}