\documentclass{beamer}
\usetheme{PegasusUCF}


%-------------------------------
\title[Fundamentals]{Fundamentals Part B}
\author{Joshua L. Eubanks}

\newcommand{\deptname}{12}%ECONOMICS

%<<<<<
\date{\vspace{-8cm}}

%--------------------------------



%%%%%%%%%%%%%%%%
\begin{document}
%%%%%%%%%%%%%%%
\begin{frame}
  \titlepage
\end{frame}

\begin{frame}{Overview}
\tableofcontents
\end{frame}


\section{Simple Production Model}

\begin{frame}{Simple Production Model}
\begin{itemize}
\item This is one of the simplest models in economics, it applies our understanding of scarcity, resources, and choice.
\item We are going to consider 2 goods to produce
\item If resources and technology are fixed, then production of one good causes a decrease in production of the other good
\end{itemize}

\begin{exampleblock}{Example: Producing Apples or Oranges}
Suppose we are deciding on how many apples or oranges to produce. How can we view this? 
\end{exampleblock}
\end{frame}


\begin{frame}{Production Possibilities Schedule}

\begin{block}{Production Possibilities Schedule}
A table that shows the possible combinations of producing two goods 
\end{block}


\begin{exampleblock}{Example: Apples and Oranges}

Consider Alex and Clara's production possibility schedules:

\begin{columns}[T] % align columns
\begin{column}{.48\textwidth}

\begin{center}
\textbf{Alex}

\begin{tabular}{ c  c }\hline
Apples & Oranges \\\hline
0 & 200 \\
100 & 100 \\
200 & 0 \\\hline
\end{tabular}

\end{center}
\end{column}%
\hfill%
\begin{column}{.48\textwidth}


\begin{center}

\textbf{Clara}

\begin{tabular}{ c  c }\hline
Apples & Oranges \\\hline
0 & 300 \\
50 & 150 \\
100 & 0 \\\hline
\end{tabular}

\end{center}

\end{column}%
\end{columns}

\end{exampleblock}
\end{frame}

\section{Production Possibilities Frontier (PPF)}

\begin{frame}{What is the PPF?}
The PPF is a graphical representation of the production possibilities schedule. Points on the PPF are considered efficient production possibilities. 

\begin{alertblock}{Note}
The next slide is intentionally left blank. Use this space to add in the PPF from my video.
\end{alertblock}

\end{frame}

\begin{frame}{PPF Visualized}

\end{frame}

\section{Opportunity Costs of Production}

\begin{frame}{Opportunity Costs of Production}
In our simple model, the opportunity cost of increasing the production of one good is the decrease in production of the second good. 

\begin{block}{Clara's opportunity cost of producing an apple}
\begin{tabular}{ c  c  | c c }\hline
Apples & Oranges & Gain (A) & Give Up (O) \\\hline
0 & 300 &  -- & --   \\
50 & 150 & 50  & 150  \\
100 & 0 & 50 & 150 \\\hline
\end{tabular}
\end{block}

It costs 3 oranges to gain 1 apple $\left(\frac{GiveUp}{Gain} = \frac{150 (oranges)}{50 (apples)} = 3 \right)$. 


\begin{alertblock}{Heads Up!}
In this example, the opportunity cost between all options is the same. This is called \textit{constant opportunitiy costs}. This may not always be the case.
\end{alertblock}


\end{frame}

\section{Absolute Advantage}

\begin{frame}{What is Absolute Advantage?}
\begin{block}{Absoulute Advantage}
The ability to produce a good using fewer inputs than another producer
\end{block}

\begin{exampleblock}{Absolute Advantage Oranges}
\begin{itemize}
\item Suppose it takes Alex 6 hours to generate one orange
\item For Clara, it takes only 1 hour to generate one orange
\item Clara has an absolute advantage in the production of Oranges
\end{itemize}
\end{exampleblock}

\begin{exampleblock}{Absolute Advantage Apples}
\begin{itemize}
\item Suppose it takes Alex 2 hours to generate one apple
\item For Clara, it takes only 1 hour to generate one apple
\item Clara has an absolute advantage in the production of apples
\end{itemize}


\end{exampleblock}

\end{frame}

\begin{frame}{Absolute Advantage Cont.}

Clara has an absolute advantage in \textbf{both} goods

\begin{itemize}
\item Absolute advantage measures the cost of a good in terms of the inputs required to produce it (Labor Hours).

\item Two individuals can gain from trade when each specializes in the good it produces at a lower opportunity cost than another producer

\item In our example, the opportunity cost of an apple is the amount of oranges that could be produced using the labor needed to produce one apple

\end{itemize}
\end{frame}

\section{Comparative Advantage}

\begin{frame}{What is Comparative Advantage?}

\begin{block}{Comparative Advantage}
The ability to produce a good or service at a lower relative opportunity cost than another producer.
\end{block}

\begin{exampleblock}{Comparing Opportunity Costs of Apples}

\begin{itemize}
\item Alex's cost of producing one apple is 1 orange
\item Clara's cost of producing one apple is 3 oranges
\item Alex's opportunity costs are lower, so he has the comparative advantage in apples

\end{itemize}
\end{exampleblock}


\end{frame}

\begin{frame}{Comparative Advantage Example Cont.}

\begin{exampleblock}{Comparing Opportunity Costs of Oranges}

\begin{itemize}
\item Alex's cost of producing one orange is 1 apple
\item Clara's cost of producing one orange is $\frac{1}{3}$ apple
\item Clara's opportunity costs are lower, so she has the comparative advantage in oranges

\end{itemize}
\end{exampleblock}
\end{frame}



\section{Specialization}

\begin{frame}{What is Specialization?}
\begin{block}{Specialization}
When a firm produces a single good or service instead of many different goods or services
\end{block}

To determine which field to specialize in, find where individual or firm has a comparative advantage. 

\begin{exampleblock}{Specialization in Apples and Oranges}
\begin{itemize}
\item Alex has the comparative advantage in apples, so he should specialize in apples
\item Clara has the comparative advantage in oranges, so she should specialize in oranges
\end{itemize}
\end{exampleblock}

\end{frame}

\begin{frame}{Overall Output Increases with Specialization}

\begin{exampleblock}

\begin{columns}[T] % align columns
\begin{column}{.48\textwidth}

\begin{center}
\textbf{No Specialization}

\begin{tabular}{| c | c c |}\hline
 & Apples & Oranges \\\hline
Alex & 100 & 100 \\
Clara & 50 & 150 \\\hline
Total & 150 & 150 \\\hline
\end{tabular}

\end{center}
\end{column}%
\hfill%
\begin{column}{.48\textwidth}


\begin{center}

\textbf{Specialization}

\begin{tabular}{ | c | c c |}\hline
 & Apples & Oranges \\\hline
Alex & 200 & 0 \\
Clara & 0 & 300 \\\hline
Total & 200 & 300 \\\hline
\end{tabular}

\end{center}

\end{column}%
\end{columns}

\end{exampleblock}

That's great and all, but what if Clara wants apples? She can \textbf{trade} her oranges for some apples.

\end{frame}

\section{Terms of Trade}

\begin{frame}{What are Terms of Trade?}

\begin{block}{Terms of Trade}
The price of one good, service, or resource in terms of another.
\end{block}

The price must be greater than the opportunity cost of the seller or producer of the item to make them better off.

\end{frame}

\begin{frame}{Alex's Terms of Trade for Apples}

\begin{exampleblock}

\begin{itemize}
\item Recall that Alex's opportunitiy cost of 1 apple is 1 orange.
\item Would it make sense for Alex to trade an apple for less than one orange? 
\item Alex would be better off producing one orange instead of trading one apple for less than one orange.
\end{itemize}

\end{exampleblock}

\end{frame}

\begin{frame}{Clara's Terms of Trade for Apples}

\begin{exampleblock}

\begin{itemize}
\item Recall that Clara's opportunitiy cost of 1 apple is 3 oranges.
\item Would it make sense for Clara to trade more than 3 oranges for 1 apple? 
\item Clara would be better off producing one apple instead of trading more than 3 oranges.
\end{itemize}
\end{exampleblock}

\end{frame}

\begin{frame}{Summary of Terms of Trade for Apples}

\begin{exampleblock}

\begin{itemize}
\item The opportunity cost of an apple for Alex (seller) was 1 orange.
\item The opportunity cost of an apple for Clara (buyer) was 3 oranges.
\item The price of an apple must be between 1 and 3 oranges
\end{itemize}
\end{exampleblock}

\end{frame}

\section{Gains of Trade}

\begin{frame}{Gains of Trade}

\begin{itemize}
\item We have shown that total output will increase with specialization
\item Additionally, we have shown that we can find the terms of trade
\item Since we can produce and trade, we end up outside the limits of the PPF
\end{itemize}

Let's look at the gain from trade whenever the terms are 1 apple for 2 oranges

\end{frame}

\begin{frame}{Clara's Gains from Trade}

\begin{exampleblock}

Clara has two choices
\begin{itemize}
\item No specialization, produces 50 apples and 150 oranges
\item Produces 300 oranges and zero oranges. She then trades 120 oranges and recieves 60 apples (terms of trade: 1 apple for 2 oranges)
\end{itemize}

\begin{tabular}{ | c | c c | }\hline
 & \textbf{No Specialization or Trade} & \textbf{Specializing and Trading} \\\hline
Clara & 50 apples 150 oranges & 60 apples 180 oranges \\\hline
\end{tabular}


\end{exampleblock}

\begin{alertblock}{Note}
The next slide is intentionally left blank. Use this space to add in the gains from trade from my video.
\end{alertblock}

\end{frame}

\begin{frame}{Gains from Trade Visualized}
\end{frame}

\section{Circular Flow Model}

\begin{frame}{What is the Circular Flow Model?}

\begin{block}{Circular Flow Model}
A model of the flow of resources, output, and monetary transactions in a simple economy
\end{block}

\begin{alertblock}{Note}
The next slide is intentionally left blank. Use this space to add in the circular flow diagram from my video.
\end{alertblock}
\end{frame}

\begin{frame}{The Circular Flow Diagram}

\end{frame}


%%%%%%%%%%%%%%
\end{document}